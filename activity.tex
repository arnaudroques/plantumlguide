% ========================================================================
% PlantUML : a free UML diagram generator
% ========================================================================
%
% (C) Copyright 2009, Arnaud Roques
%
% Project Info:  http://plantuml.sourceforge.net
% 
% This file is part of PlantUML.
%
% PlantUML is free software; you can redistribute it and/or modify it
% under the terms of the GNU General Public License as published by
% the Free Software Foundation, either version 3 of the License, or
% (at your option) any later version.
%
% PlantUML distributed in the hope that it will be useful, but
% WITHOUT ANY WARRANTY; without even the implied warranty of MERCHANTABILITY
% or FITNESS FOR A PARTICULAR PURPOSE. See the GNU Lesser General Public
% License for more details.
%
% You should have received a copy of the GNU General Public
% License along with this library; if not, write to the Free Software
% Foundation, Inc., 51 Franklin Street, Fifth Floor, Boston, MA  02110-1301,
% USA.
%
% [Java is a trademark or registered trademark of Sun Microsystems, Inc.
% in the United States and other countries.]
%
% Original Author:  Arnaud Roques
% 

\section{Activity Diagram}

% ========================================================================
\subsection{Simple Activity}

\begin{description}
\item
You can use \texttt{(*)} for the starting point and ending point of the activity
diagram.

\item
Use \texttt{-->} for arrows.

\end{description}

Example:

\begin{lstlisting}
@startuml
(*) --> "First Activity"
"First Activity" --> (*)
@enduml
\end{lstlisting}

\begin{center}
\includegraphics[width=40mm]{activity.png}
\end{center}

% ========================================================================
\subsection{Label on arrows}

\begin{description}
\item By default, an arrow starts at the last used activity.
\item You can put a label on a arrow using brackets \texttt{[} and \texttt{]}
just after the arrow definition.
\end{description}


\begin{lstlisting}
@startuml
(*) --> "First Activity"
-->[You can put also labels] "Second Activity"
--> (*)
@enduml
\end{lstlisting}
\begin{center}
\includegraphics[width=55mm]{activity_001.png}
\end{center}

% ========================================================================
\newpage  \subsection{Changing arrow direction}

You can use \texttt{->} for horizontal arrows. It is possible to force arrow's
direction using the following syntax:

\begin{itemize}
\item \texttt{-down->} (default arrow)
\item \texttt{-right->} or \texttt{->}
\item \texttt{-left->}
\item \texttt{-up->}
\end{itemize}

\begin{lstlisting}
@startuml
(*) -up-> "First Activity"
-right-> "Second Activity"
--> "Third Activity"
-left-> (*)
@enduml
\end{lstlisting}
\begin{center}
\includegraphics[width=80mm]{activity_002.png}
\end{center}

You can shorten the arrow by using only the first character of the direction
(for example, \texttt{-d-} instead of \texttt{-down-}) or the two first
characters (\texttt{-do-}).

Please note that you should not abuse this functionnality : \textit{GraphViz}
gives usually good results without tweaking.

% ========================================================================
\newpage \subsection{Branches}
\begin{description}
\item You can use \texttt{if/then/else} keywords to define branches.
\end{description}

\begin{lstlisting}
@startuml
(*) --> "Initialisation"

if "Some Test" then
  -->[true] "Some Activity"
  --> "Another activity"
  -right-> (*)
else
  ->[false] "Something else"
  -->[Ending process] (*)
endif
@enduml
\end{lstlisting}
\begin{center}
\includegraphics[width=70mm]{activity_003.png}
\end{center}


% ========================================================================
\newpage \subsection{More on Branches}
\begin{description}
\item By default, a branch is connected to the last defined activity,
but it is possible to override this and to define a link with the
\texttt{if} keywords.
\item It is also possible to nest branches.
\end{description}

\begin{lstlisting}
@startuml
(*) --> if "Some Test" then

  -->[true] "activity 1"
  
  if "" then
    -> "activity 3" as a3
  else
    if "Other test" then
      -left-> "activity 5"
    else
      --> "activity 6"
    endif
  endif
  
else

  ->[false] "activity 2"
  
endif

a3 --> if "last test" then
  --> "activity 7"
else
  -> "activity 8"
endif
@enduml
\end{lstlisting}
\begin{center}
\includegraphics[width=110mm]{activity_004.png}
\end{center}



% ========================================================================
\newpage \subsection{Synchronization}

You can use "\texttt{=== code ===}" to display synchronization bars.

\begin{lstlisting}
@startuml

(*) --> ===B1=== 
--> "Parallel Activity 1"
--> ===B2===

===B1=== --> "Parallel Activity 2"
--> ===B2===

--> (*)

@enduml
\end{lstlisting}
\begin{center}
\includegraphics[width=90mm]{activity_005.png}
\end{center}



% ========================================================================
\newpage \subsection{Long activity description}

When you declare activities, you can span on several lines the description text.
You can also add \texttt{\textbackslash n } in the description.
It is also possible to use few html tags like :

\begin{itemize}
\item \texttt{$<$b$>$}
\item \texttt{$<$i$>$}
\item \texttt{$<$font size="nn"$>$} or \texttt{$<$size:nn$>$} to change font
size
\item \texttt{$<$font color="\#AAAAAA"$>$} or \texttt{$<$font
color="colorName"$>$}
\item \texttt{$<$color:\#AAAAAA$>$} or \texttt{$<$color:colorName$>$}
\item \texttt{$<$img:file.png$>$} to include an image
\end{itemize}

You can also give a short code to the activity with the as keyword.
This code can be used latter in the diagram description.

\begin{lstlisting}
@startuml
(*) -left-> "this <size:20>activity</size>
	is <b>very</b> <color:red>long</color>
	and defined on several lines
	that contains many <i>text</i>" as A1

-up-> "Another activity\n on several lines"

A1 --> "Short activity <img:sourceforge.jpg>"
@enduml
\end{lstlisting}
\begin{center}
\includegraphics[width=75mm]{activity_006.png}
\end{center}



% ========================================================================
\newpage \subsection{Notes}

You can add notes on a activity using the commands:
\begin{itemize}
  \item \texttt{note left},
  \item \texttt{note right},
  \item \texttt{note top},
  \item \texttt{note bottom},
\end{itemize}
just after the description of the activity you want to note.
If you want to put a note on the starting point, define the note at the very
beginning of the diagram description.

You can also have a note on several lines, using the \texttt{end note} keywords.

\begin{lstlisting}
@startuml

(*) --> "Some Activity"
note right: This activity has to be defined
"Some Activity" --> (*)
note left
 This note is on
 several lines
end note
@enduml
\end{lstlisting}
\begin{center}
\includegraphics[width=90mm]{activity_007.png}
\end{center}

% ========================================================================
\newpage \subsection{Partition}

\begin{description}
\item You can define a partition using the \texttt{partition} keyword, and
optionally declare a background color for your partition (using a html color code or name).
\item When you declare activities, they are automatically put in the last used partition.
\end{description}

You can close the partition definition using the \texttt{end partition} keyword.

\begin{lstlisting}
@startuml

partition Conductor
(*) --> "Climbs on Platform"
--> === S1 ===
--> Bows
end partition

partition Audience #LightSkyBlue
=== S1 === --> Applauds

partition Conductor
Bows --> === S2 ===
--> WavesArmes
Applauds --> === S2 ===
end partition

partition Orchestra #CCCCEE
WavesArmes --> Introduction
--> "Play music"
end partition

@enduml
\end{lstlisting}
\begin{center}
\includegraphics[width=80mm]{activity_008.png}
\end{center}
		
% ========================================================================
\newpage \subsection{Title the diagram}

The \texttt{title} keywords is used to put a title.
You can use \texttt{title} and \texttt{end title} keywords for a longer title,
as in sequence diagrams.

\begin{lstlisting}
@startuml
title Simple example\nof title 

(*) --> "First activity"
--> (*)
@enduml
\end{lstlisting}
\begin{center}
\includegraphics[width=40mm]{activity_009.png}
\end{center}

% ========================================================================
\newpage \subsection{Skinparam}

You can use the \texttt{skinparam} command to change colors and fonts for the
drawing. You can use this command :

\begin{itemize}
\item In the diagram definition, like any other commands,
\item In an included file,
\item In a configuration file, provided in the command line or the ANT task.
\end{itemize}

\begin{lstlisting}
@startuml

skinparam backgroundColor #AAFFFF
skinparam activity {
	StartColor red
	BarColor SaddleBrown 
	EndColor Silver
	BackgroundColor Peru
	BorderColor Peru
	FontName Impact
}

(*) --> "Climbs on Platform"
--> === S1 ===
--> Bows
--> === S2 ===
--> WavesArmes
--> (*)

@enduml
\end{lstlisting}
\begin{center}
\includegraphics[width=50mm]{activity_010.png}
\end{center}


% ========================================================================
\newpage \subsection{Complete example}

\begin{lstlisting}
@startuml
'http://click.sourceforge.net/images/activity-diagram-small.png
title Servlet Container

(*) --> "ClickServlet.handleRequest()"
--> "new Page"

if "Page.onSecurityCheck" then
  ->[true] "Page.onInit()"
  
  if "isForward?" then
   ->[no] "Process controls"
   
   if "continue processing?" then
     -->[yes] ===RENDERING===
   else
     -->[no] ===REDIRECT_CHECK===
   endif
   
  else
   -->[yes] ===RENDERING===
  endif
  
  if "is Post?" then
    -->[yes] "Page.onPost()"
    --> "Page.onRender()" as render
    --> ===REDIRECT_CHECK===
  else
    -->[no] "Page.onGet()"
    --> render
  endif
  
else
  -->[false] ===REDIRECT_CHECK===
endif

if "Do redirect?" then
 ->[yes] "redirect request"
 --> ==BEFORE_DESTROY===
else
 if "Do Forward?" then
  -left->[yes] "Forward request"
  --> ==BEFORE_DESTROY===
 else
  -right->[no] "Render page template"
  --> ==BEFORE_DESTROY===
 endif
endif

--> "Page.onDestroy()"
-->(*)

@enduml
\end{lstlisting}
\begin{center}
\includegraphics[width=130mm]{activity_011.png}
\end{center}
