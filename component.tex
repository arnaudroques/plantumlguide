% ========================================================================
% PlantUML : a free UML diagram generator
% ========================================================================
%
% (C) Copyright 2009, Arnaud Roques
%
% Project Info:  http://plantuml.sourceforge.net
% 
% This file is part of PlantUML.
%
% PlantUML is free software; you can redistribute it and/or modify it
% under the terms of the GNU General Public License as published by
% the Free Software Foundation, either version 3 of the License, or
% (at your option) any later version.
%
% PlantUML distributed in the hope that it will be useful, but
% WITHOUT ANY WARRANTY; without even the implied warranty of MERCHANTABILITY
% or FITNESS FOR A PARTICULAR PURPOSE. See the GNU Lesser General Public
% License for more details.
%
% You should have received a copy of the GNU General Public
% License along with this library; if not, write to the Free Software
% Foundation, Inc., 51 Franklin Street, Fifth Floor, Boston, MA  02110-1301,
% USA.
%
% [Java is a trademark or registered trademark of Sun Microsystems, Inc.
% in the United States and other countries.]
%
% Original Author:  Arnaud Roques
% 

\section{Component Diagram}

% ========================================================================
\subsection{Components}

\begin{description}
\item Components must be bracketed.
\item You can also use the \texttt{component} keyword to defines a component.
\item And you can define an alias, using the \texttt{as} keyword.
\item This alias will be used latter, when defining relations.
\end{description}

\begin{lstlisting}
@startuml

[First component]
[Another component] as Comp2  
component Comp3
component [Last\ncomponent] as Comp4

@enduml
\end{lstlisting}
\begin{center}
\includegraphics[width=125mm]{component.png}
\end{center}

% ========================================================================
\newpage \subsection{Interfaces}

\begin{description}
\item Interface can be defined using the "\texttt{()}" symbol (because this
looks like a circle).
\item You can also use the \texttt{interface} keyword to defines a usecase.
\item And you can define an alias, using the \texttt{as} keyword.
\item This alias will be used latter, when defining relations.
\item We will see latter that interface declaration is optional.
\end{description}

\begin{lstlisting}
@startuml

() "First Interface"
() "Another interface" as Interf2
interface Interf3
interface "Last\ninterface" as Interf4

@enduml
\end{lstlisting}

\begin{center}
\includegraphics[width=105mm]{component_001.png}
\end{center}


% ========================================================================
\newpage \subsection{Basic example}
Links between elements are made using combinaisons of dotted line
"\texttt{..}", straight line "\texttt{--}", and arrows "\texttt{-->}" symbols.

\begin{lstlisting}
@startuml

DataAccess - [First Component] 
[First Component] ..> HTTP : use

@enduml
\end{lstlisting}

\begin{center}
\includegraphics[width=65mm]{component_002.png}
\end{center}

% ========================================================================
\newpage \subsection{Using notes}

You can use the
\begin{itemize}
  \item \texttt{note left of}, 
  \item \texttt{note right of},
  \item \texttt{note top of},
  \item \texttt{note bottom of},  
\end{itemize}
keywords to define notes related to a single object.

A note can  also be defined alone with the \texttt{note} keywords, then linked
to other objects using the "\texttt{..}" symbol.

\begin{lstlisting}
@startuml

interface "Data Access" as DA

DA - [First Component] 
[First Component] ..> HTTP : use

note left of HTTP : Web Service only

note right of [First Component]
  A note can also
  be on several lines
end note

@enduml
\end{lstlisting}

\begin{center}
\includegraphics[width=105mm]{component_003.png}
\end{center}

% ========================================================================
\newpage \subsection{Grouping Components}

You can use the \texttt{package} keyword to group components and interfaces
together.

\begin{lstlisting}
@startuml

package "Some Group" {
  HTTP - [First Component]
  [Another Component]
}
 
package "Other Groups" {
  FTP - [Second Component]
  [First Component] --> FTP
}
@enduml
\end{lstlisting}

\begin{center}
\includegraphics[width=120mm]{component_004.png}
\end{center}

% ========================================================================
\newpage \subsection{Changing arrows direction}

By default, links between classes have two dashes \texttt{--}
 and are verticaly oriented. It is possible to use horizontal link by putting a
 single dash (or dot) like this:

\begin{lstlisting}
@startuml
[Component] --> Interface1
[Component] -> Interface2
@enduml
\end{lstlisting}

\begin{center}
\includegraphics[width=50mm]{component_005.png}
\end{center}

You can also change directions by reversing the link:

\begin{lstlisting}
@startuml
Interface1 <-- [Component]
Interface2 <- [Component]
@enduml
\end{lstlisting}

\begin{center}
\includegraphics[width=50mm]{component_006.png}
\end{center}

It is also possible to change arrow direction by adding \texttt{left},
\texttt{right}, \texttt{up} or \texttt{down} keywords inside the arrow:

\begin{lstlisting}
@startuml
[Component] -left-> left 
[Component] -right-> right 
[Component] -up-> up
[Component] -down-> down
@enduml
\end{lstlisting}

\begin{center}
\includegraphics[width=75mm]{component_007.png}
\end{center}

You can shorten the arrow by using only the first character of the direction
(for example, \texttt{-d-} instead of \texttt{-down-}) or the two first
characters (\texttt{-do-}).

Please note that you should not abuse this functionnality : \textit{GraphViz}
gives usually good results without tweaking.

% ========================================================================
\subsection{Title the diagram}

The \texttt{title} keywords is used to put a title.
You can use \texttt{title} and \texttt{end title} keywords for a longer title,
as in sequence diagrams.

\begin{lstlisting}
@startuml
title Very simple component\ndiagram

interface "Data Access" as DA

DA - [First Component] 
[First Component] ..> HTTP : use

@enduml
\end{lstlisting}

\begin{center}
\includegraphics[width=70mm]{component_008.png}
\end{center}


% ========================================================================
\newpage \subsection{Skinparam}

You can use the \texttt{skinparam} command to change colors and fonts for the
drawing. You can use this command :

\begin{itemize}
\item In the diagram definition, like any other commands,
\item In an included file,
\item In a configuration file, provided in the command line or the ANT task.
\end{itemize}

\begin{lstlisting}
@startuml
skinparam component {
  FontSize 13
  InterfaceBackgroundColor RosyBrown
  InterfaceBorderColor orange
  BackgroundColor<< Apache >> Red
  BorderColor<< Apache >> #FF6655
  FontName Courier
  BorderColor black
  BackgroundColor gold
  ArrowFontName Impact
  ArrowColor #FF6655
  ArrowFontColor #777777
}

() "Data Access" as DA

DA - [First Component] 
[First Component] ..> () HTTP : use
HTTP - [Web Server] << Apache >>
@enduml
\end{lstlisting}

\begin{center}
\includegraphics[width=75mm]{component_009.png}
\end{center}
				