% ========================================================================
% PlantUML : a free UML diagram generator
% ========================================================================
%
% (C) Copyright 2009, Arnaud Roques
%
% Project Info:  http://plantuml.sourceforge.net
% 
% This file is part of PlantUML.
%
% PlantUML is free software; you can redistribute it and/or modify it
% under the terms of the GNU General Public License as published by
% the Free Software Foundation, either version 3 of the License, or
% (at your option) any later version.
%
% PlantUML distributed in the hope that it will be useful, but
% WITHOUT ANY WARRANTY; without even the implied warranty of MERCHANTABILITY
% or FITNESS FOR A PARTICULAR PURPOSE. See the GNU Lesser General Public
% License for more details.
%
% You should have received a copy of the GNU General Public
% License along with this library; if not, write to the Free Software
% Foundation, Inc., 51 Franklin Street, Fifth Floor, Boston, MA  02110-1301,
% USA.
%
% [Java is a trademark or registered trademark of Sun Microsystems, Inc.
% in the United States and other countries.]
%
% Original Author:  Arnaud Roques
% 

\section{Objects Diagram}

% ========================================================================
\subsection{Definition of objects}

You define instance of objects using the \texttt{object} keywords.

\begin{lstlisting}
@startuml
object firstObject
object "My Second Object" as o2

@enduml
\end{lstlisting}
\begin{center}
\includegraphics[width=60mm]{object.png}
\end{center}

% ========================================================================
\subsection{Relations between objects}

\begin{description}
\item Relations between objects are defined using the following symbols : 

\begin{tabular}{|l|l|l|} \hline
Extension & \texttt{<|--} & 
\includegraphics[width=6mm]{img/extends01.png}

\\ \hline
Composition & \texttt{*--} &
\includegraphics[width=7mm]{img/sym03.png}

\\ \hline
Agregation & \texttt{o--} &
\includegraphics[width=7mm]{img/sym01.png}

\\ \hline
\end{tabular}

\item It is possible to replace "\texttt{--}" by "\texttt{..}" to have a dotted
line.

\item Knowing thoses rules, it is possible to draw the following drawings: 
\end{description}

\begin{lstlisting}
@startuml
object Object01
object Object02
object Object03
object Object04
object Object05
object Object06
object Object07
object Object08

Object01 <|-- Object02
Object03 *-- Object04
Object05 o-- "4" Object06
Object07 .. Object08 : some labels
@enduml
\end{lstlisting}
\begin{center}
\includegraphics[width=100mm]{object_001.png}
\end{center}

% ========================================================================
\newpage \subsection{Adding fields}

To declare fields, you can use the symbol "\texttt{:}" followed by the field's
name.

\begin{lstlisting}
@startuml

object user

user : name = "Dummy"
user : id = 123


@enduml
\end{lstlisting}
\begin{center}
\includegraphics[width=25mm]{object_002.png}
\end{center}

It is also possible to ground between brackets \texttt{\{\}} all fields.

\begin{lstlisting}
@startuml
object user {
  name = "Dummy"
  id = 123
}
@enduml
\end{lstlisting}
\begin{center}
\includegraphics[width=25mm]{object_003.png}
\end{center}



	