% ========================================================================
% PlantUML : a free UML diagram generator
% ========================================================================
%
% (C) Copyright 2009, Arnaud Roques
%
% Project Info:  http://plantuml.sourceforge.net
% 
% This file is part of PlantUML.
%
% PlantUML is free software; you can redistribute it and/or modify it
% under the terms of the GNU General Public License as published by
% the Free Software Foundation, either version 3 of the License, or
% (at your option) any later version.
%
% PlantUML distributed in the hope that it will be useful, but
% WITHOUT ANY WARRANTY; without even the implied warranty of MERCHANTABILITY
% or FITNESS FOR A PARTICULAR PURPOSE. See the GNU Lesser General Public
% License for more details.
%
% You should have received a copy of the GNU General Public
% License along with this library; if not, write to the Free Software
% Foundation, Inc., 51 Franklin Street, Fifth Floor, Boston, MA  02110-1301,
% USA.
%
% [Java is a trademark or registered trademark of Sun Microsystems, Inc.
% in the United States and other countries.]
%
% Original Author:  Arnaud Roques
% 

\section{Use Case Diagram}

% ========================================================================
\subsection{Usecases}

\begin{description}
\item Use cases are enclosed using between parentheses (because two parentheses
looks like an oval).
\item You can also use the \texttt{usecase} keyword to define a usecase.
\item And you can define an alias, using the \texttt{as} keyword. This alias
will be used latter, when defining relations.
\end{description}

\begin{lstlisting}
@startuml

(First usecase)
(Another usecase) as (UC2)  
usecase UC3
usecase (Last\nusecase) as UC4

@enduml
\end{lstlisting}
\begin{center}
\includegraphics[width=140mm]{usecase.png}
\end{center}

% ========================================================================
\newpage \subsection{Actors}

\begin{description}
\item Actor are enclosed using between two points.
\item You can also use the \texttt{actor} keyword to define an actor.
\item And you can define an alias, using the \texttt{as} keyword. This alias will be used latter, when defining relations.
\item We will see than actor definitions is optional.
\end{description}


\begin{lstlisting}
@startuml

:First Actor:
:Another\nactor: as Men2  
actor Men3
actor :Last actor: as Men4

@enduml
\end{lstlisting}
\begin{center}
\includegraphics[width=100mm]{usecase_001.png}
\end{center}


% ========================================================================
\newpage \subsection{Basic example}

\begin{description}
\item To link actors and use cases, the arrow "\texttt{-->}" is used.
\item The more dashes "\texttt{-}" in the arrow, the longer the arrow. 
\item You can add a label on the arrow, by adding a "\texttt{:}" character in
the arrow definition.
\item In this example, you see that \textit{User} has not been defined before,
and is implicitly defined as an actor.
\end{description}

\begin{lstlisting}
@startuml

User -> (Start)
User --> (Use the application) : A small label

:Main Admin: ---> (Use the application) : This is\nyet another\nlabel

@enduml
\end{lstlisting}
\begin{center}
\includegraphics[width=95mm]{usecase_002.png}
\end{center}

% ========================================================================
\newpage \subsection{Extension}

\begin{description}
\item If one actor/use case extends another one, you can use the symbol
\texttt{<|--} (which stands for \includegraphics[width=6mm]{img/extends01.png}).
\item As for smiley, when you turn your head, you will see the symbol
\includegraphics[width=2mm]{img/extends02.png}
\end{description}
  	
\begin{lstlisting}
@startuml
:Main Admin: as Admin
(Use the application) as (Use)

User <|-- Admin
(Start) <|-- (Use)

@enduml
\end{lstlisting}
\begin{center}
\includegraphics[width=85mm]{usecase_003.png}
\end{center}

% ========================================================================
\newpage \subsection{Using notes}

You can use the :
\begin{itemize}
  \item \texttt{note left of},
  \item \texttt{note right of}, 
  \item \texttt{note top of},
  \item \texttt{note bottom of}
\end{itemize}
keywords to define notes related to a single object.
A note can be also define alone with the note keywords, then
linked to other objects using the \texttt{..} symbol.

\begin{lstlisting}
@startuml
:Main Admin: as Admin
(Use the application) as (Use)

User -> (Start)
User --> (Use)

Admin ---> (Use)

note right of Admin : This is an example.

note right of (Use)
  A note can also
  be on several lines
end note

note "This note is connected\nto several objects." as N2
(Start) .. N2
N2 .. (Use)
@enduml
\end{lstlisting}
\begin{center}
\includegraphics[width=130mm]{usecase_004.png}
\end{center}

% ========================================================================
\newpage \subsection{Stereotypes}

You can add stereotypes while defining actors and use cases using "\texttt{<<}" and "\texttt{>>}".

\begin{lstlisting}
@startuml
User << Human >>
:Main Database: as MySql << Application >>
(Start) << One Shot >>
(Use the application) as (Use) << Main >>

User -> (Start)
User --> (Use)

MySql --> (Use)

@enduml
\end{lstlisting}
\begin{center}
\includegraphics[width=120mm]{usecase_005.png}
\end{center}

% ========================================================================
\newpage \subsection{Changing arrows direction}

By default, links between classes have two dashes \texttt{--} and are verticaly
oriented. It is possible to use horizontal link by putting a single dash (or dot) like this:
\begin{lstlisting}
@startuml
:user: --> (Use case 1)
:user: -> (Use case 2)
@enduml
\end{lstlisting}
\begin{center}
\includegraphics[width=60mm]{usecase_006.png}
\end{center}

You can also change directions by reversing the link:
\begin{lstlisting}
@startuml
(Use case 1) <.. :user:
(Use case 2) <- :user:
@enduml
\end{lstlisting}
\begin{center}
\includegraphics[width=60mm]{usecase_007.png}
\end{center}

It is also possible to change arrow direction by adding \texttt{left},
\texttt{right}, \texttt{up} or \texttt{down} keywords inside the arrow:
\begin{lstlisting}
@startuml
:user: -left-> (dummyLeft) 
:user: -right-> (dummyRight) 
:user: -up-> (dummyUp)
:user: -down-> (dummyDown)
@enduml
\end{lstlisting}
\begin{center}
\includegraphics[width=100mm]{usecase_008.png}
\end{center}

You can shorten the arrow by using only the first character of the direction
(for example, \texttt{-d-} instead of \texttt{-down-}) or the two first
characters (\texttt{-do-}).

Please note that you should not abuse this functionnality : \textit{GraphViz}
gives usually good results without tweaking.

% ========================================================================
\subsection{Title the diagram}

\begin{description}
\item The \texttt{title} keywords is used to put a title.
\item You can use \texttt{title} and \texttt{end title} keywords for a longer
title, as in sequence diagrams.
\end{description}

\begin{lstlisting}
@startuml
title Simple <b>Usecase</b>\nwith one actor

usecase (Use the application) as (Use)
User -> (Use)

@enduml
\end{lstlisting}
\begin{center}
\includegraphics[width=8cm]{usecase_009.png}
\end{center}

% ========================================================================
\newpage \subsection{Left to right direction}

The general default behaviour when building diagram is top to bottom.


\begin{lstlisting}
@startuml
'default
top to bottom direction
user1 --> (Usecase 1)
user2 --> (Usecase 2)

@enduml
\end{lstlisting}
\begin{center}
\includegraphics[width=70mm]{usecase_010.png}
\end{center}

You may change to left to right using the \texttt{left to right direction} command. The result is often better with this direction.
\begin{lstlisting}
@startuml

left to right direction
user1 --> (Usecase 1)
user2 --> (Usecase 2)

@enduml
\end{lstlisting}
\begin{center}
\includegraphics[width=70mm]{usecase_011.png}
\end{center}

% ========================================================================
\newpage \subsection{Skinparam}

You can use the \texttt{skinparam} command to change colors and fonts for the drawing.
You can use this command :

\begin{itemize}
\item In the diagram definition, like any other commands,
\item In an included file,
\item In a configuration file, provided in the command line or the ANT task.
\end{itemize}


\begin{lstlisting}
@startuml
skinparam usecase {
	BackgroundColor DarkSeaGreen
	BorderColor DarkSlateGray

	BackgroundColor<< Main >> YellowGreen
	BorderColor<< Main >> YellowGreen
	
	ArrowColor Olive
	ActorBorderColor black
	ActorFontName Courier

	ActorBackgroundColor<< Human >> Gold
}

User << Human >>
:Main Database: as MySql << Application >>
(Start) << One Shot >>
(Use the application) as (Use) << Main >>

User -> (Start)
User --> (Use)

MySql --> (Use)

@enduml
\end{lstlisting}
\begin{center}
\includegraphics[width=110mm]{usecase_012.png}
\end{center}
