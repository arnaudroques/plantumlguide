% ========================================================================
% PlantUML : a free UML diagram generator
% ========================================================================
%
% (C) Copyright 2009, Arnaud Roques
%
% Project Info:  http://plantuml.sourceforge.net
% 
% This file is part of PlantUML.
%
% PlantUML is free software; you can redistribute it and/or modify it
% under the terms of the GNU General Public License as published by
% the Free Software Foundation, either version 3 of the License, or
% (at your option) any later version.
%
% PlantUML distributed in the hope that it will be useful, but
% WITHOUT ANY WARRANTY; without even the implied warranty of MERCHANTABILITY
% or FITNESS FOR A PARTICULAR PURPOSE. See the GNU Lesser General Public
% License for more details.
%
% You should have received a copy of the GNU General Public
% License along with this library; if not, write to the Free Software
% Foundation, Inc., 51 Franklin Street, Fifth Floor, Boston, MA  02110-1301,
% USA.
%
% [Java is a trademark or registered trademark of Sun Microsystems, Inc.
% in the United States and other countries.]
%
% Original Author:  Arnaud Roques
% 
% ========================================================================
% PlantUML : a free UML diagram generator
% ========================================================================
%
% (C) Copyright 2009, Arnaud Roques
%
% Project Info:  http://plantuml.sourceforge.net
% 
% This file is part of PlantUML.
%
% PlantUML is free software; you can redistribute it and/or modify it
% under the terms of the GNU General Public License as published by
% the Free Software Foundation, either version 3 of the License, or
% (at your option) any later version.
%
% PlantUML distributed in the hope that it will be useful, but
% WITHOUT ANY WARRANTY; without even the implied warranty of MERCHANTABILITY
% or FITNESS FOR A PARTICULAR PURPOSE. See the GNU Lesser General Public
% License for more details.
%
% You should have received a copy of the GNU General Public
% License along with this library; if not, write to the Free Software
% Foundation, Inc., 51 Franklin Street, Fifth Floor, Boston, MA  02110-1301,
% USA.
%
% [Java is a trademark or registered trademark of Sun Microsystems, Inc.
% in the United States and other countries.]
%
% Original Author:  Arnaud Roques
% 

\section{Common commands}

% ========================================================================
\subsection{Footer and header} 

\begin{description}
\item You can use the commands \texttt{header} or \texttt{footer} to add a
footer or a header on any generated diagram. 
\item You can optionally specify if you want a \texttt{center}, \texttt{left} or
\texttt{right} footer/header, by adding a keywork. 
\item As for title, it is possible to define a header or a footer on several lines. 
\item It is also possible to put some HTML into the header or footer 
\end{description}


\begin{lstlisting}
@startuml
Alice -> Bob: Authentication Request 

header 
<font color=red>Warning:</font> This is a demonstration diagram.
Do not use in production.
endheader 

center footer Generated for demonstration 
@enduml 
\end{lstlisting}

\begin{center}
\includegraphics[width=50mm]{common.png}
\end{center}

% ========================================================================
\newpage \subsection{Zoom}

You can use the scale command to zoom the generated image.
You can use either a number or a fraction to define the scale factor.
You can also specify either width or height (in pixel).
And you can also give both width and height : the image is scaled to fit inside the specified dimension.

\begin{itemize}
\item \texttt{scale 1.5}, 
\item \texttt{scale 2/3}, 
\item \texttt{scale 200 width}, 
\item \texttt{scale 200 height}, 
\item \texttt{scale 200*100} 
\end{itemize}

\begin{lstlisting}
@startuml
scale 180*90
Bob->Alice : hello
@enduml 

\end{lstlisting}

\begin{center}
\includegraphics[width=30mm]{common_001.png}
\end{center}


% ========================================================================
\newpage \subsection{Rotation} 

Sometimes, and especially for printing, you may want to rotate the generated image, so that it fits 
better in the page. 
You can use the \texttt{rotate} command for this. 

\begin{lstlisting}
@startuml
rotate

title Simple Usecase\nwith one actor 
"Use the application" as (Use)
User -> (Use) 

@enduml 

\end{lstlisting}

\begin{center}
\includegraphics[width=30mm]{common_002.png}
\end{center}

