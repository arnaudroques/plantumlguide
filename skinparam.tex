% ========================================================================
% PlantUML : a free UML diagram generator
% ========================================================================
%
% (C) Copyright 2009, Arnaud Roques
%
% Project Info:  http://plantuml.sourceforge.net
% 
% This file is part of PlantUML.
%
% PlantUML is free software; you can redistribute it and/or modify it
% under the terms of the GNU General Public License as published by
% the Free Software Foundation, either version 3 of the License, or
% (at your option) any later version.
%
% PlantUML distributed in the hope that it will be useful, but
% WITHOUT ANY WARRANTY; without even the implied warranty of MERCHANTABILITY
% or FITNESS FOR A PARTICULAR PURPOSE. See the GNU Lesser General Public
% License for more details.
%
% You should have received a copy of the GNU General Public
% License along with this library; if not, write to the Free Software
% Foundation, Inc., 51 Franklin Street, Fifth Floor, Boston, MA  02110-1301,
% USA.
%
% [Java is a trademark or registered trademark of Sun Microsystems, Inc.
% in the United States and other countries.]
%
% Original Author:  Arnaud Roques
% 

\section{Changing fonts and colors}

% ========================================================================
\subsection{Usage}

You can change colors and font of the drawing using the \texttt{skinparam} command. 
Example: 

\begin{lstlisting}
skinparam backgroundColor yellow 
\end{lstlisting}


You can use this command : 

\begin{itemize}
\item In the diagram definition, like any other commands, 
\item In an included file (see \textit{Preprocessing}), 
\item In a configuration file, provided in the command line or the ANT task.
\end{itemize}

% ========================================================================
\subsection{Nested}
To avoid repetition, it is possible to nest definition. So the following definition :

\begin{lstlisting}
skinparam xxxxParam1 value1
skinparam xxxxParam2 value2
skinparam xxxxParam3 value3
skinparam xxxxParam4 value4
\end{lstlisting}
		

is strictly equivalent to:

\begin{lstlisting}
skinparam xxxx {
  Param1 value1
  Param2 value2
  Param3 value3
  Param4 value4
}
\end{lstlisting}
		



 
% ========================================================================
\newpage \subsection{Color}

You can use either standard color name or RGB code.
\vskip 12mm

\newcommand{\firstCell}{\texttt}
 
\begin{center}
\begin{tabular}{|l|c|c|l|} \hline

\textbf{Parameter name} & \textbf{Default} & \textbf{Color} & \textbf{Comment}
\\
& \textbf{Value} & &

\\ \hline

\firstCell{backgroundColor} & \footnotesize{white}
& & \footnotesize{Background of the page}

\\ \hline

\firstCell{activityArrowColor} & \footnotesize{\#A80036} &
\cellcolor[HTML]{A80036} &
\footnotesize{Color of arrows in activity diagrams}

\\ \hline

\firstCell{activityBackgroundColor} & \footnotesize{\#FEFECE} &
\cellcolor[HTML]{FEFECE} &
\footnotesize{Background of activities}

\\ \hline
 
\firstCell{activityBorderColor} & \footnotesize{\#A80036} &
\cellcolor[HTML]{A80036} &
\footnotesize{Color of activity borders}

\\ \hline
 
\firstCell{activityStartColor} & \footnotesize{black} &
\cellcolor[HTML]{000000} &
\footnotesize{Starting circle in activity diagrams}
 
\\ \hline
 
\firstCell{activityEndColor} & \footnotesize{black} &
\cellcolor[HTML]{000000} &
\footnotesize{Ending circle in activity diagrams}

\\ \hline
 
\firstCell{activityBarColor} & \footnotesize{black} &
\cellcolor[HTML]{000000} &
\footnotesize{Synchronization bar in activity diagrams}
 
\\ \hline
 
\firstCell{usecaseArrowColor} & \footnotesize{\#A80036} &
\cellcolor[HTML]{A80036} &
\footnotesize{Color of arrows in usecase diagrams}

\\ \hline
 
\firstCell{usecaseActorBackgroundColor} & \footnotesize{\#FEFECE} &
\cellcolor[HTML]{FEFECE} &
\footnotesize{Head's color of actor in usecase diagrams}
 
\\ \hline
 
\firstCell{usecaseActorBorderColor} & \footnotesize{\#A80036} &
\cellcolor[HTML]{A80036} &
\footnotesize{Color of actor borders in usecase diagrams}

\\ \hline
 
\firstCell{usecaseBackgroundColor} & \footnotesize{\#FEFECE} &
\cellcolor[HTML]{FEFECE} &
\footnotesize{Background of usecases}

\\ \hline
 
\firstCell{usecaseBorderColor} & \footnotesize{\#A80036} &
\cellcolor[HTML]{A80036} &
\footnotesize{Color of usecase borders in usecase diagrams}
 
\\ \hline
 
\firstCell{classArrowColor} & \footnotesize{\#A80036} &
\cellcolor[HTML]{A80036} &
\footnotesize{Color of arrows in class diagrams}

\\ \hline
 
\firstCell{classBackgroundColor} & \footnotesize{\#FEFECE} &
\cellcolor[HTML]{FEFECE} &
\footnotesize{Background of classes/interface/enum in class diagrams}
 
\\ \hline
 
\firstCell{classBorderColor} & \footnotesize{\#A80036} &
\cellcolor[HTML]{A80036} &
\footnotesize{Borders of classes/interface/enum in class diagrams}
 
\\ \hline
 
\firstCell{packageBackgroundColor} & \footnotesize{\#FEFECE} &
\cellcolor[HTML]{FEFECE} &
\footnotesize{Background of packages in class diagrams}
 
\\ \hline
 
\firstCell{packageBorderColor} & \footnotesize{\#A80036} &
\cellcolor[HTML]{A80036} &
\footnotesize{Borders of packages in class diagrams}

\\ \hline
 
\firstCell{stereotypeCBackgroundColor} & \footnotesize{\#ADD1B2} &
\cellcolor[HTML]{ADD1B2} &
\footnotesize{Background of class spots in class diagrams}

\\ \hline
 
\firstCell{stereotypeABackgroundColor} & \footnotesize{\#A9DCDF} &
\cellcolor[HTML]{A9DCDF} &
\footnotesize{Background of abstract class spots in class diagrams}
 
\\ \hline
 
\firstCell{stereotypeIBackgroundColor} & \footnotesize{\#B4A7E5} &
\cellcolor[HTML]{B4A7E5} &
\footnotesize{Background of interface spots in class diagrams}
 
\\ \hline
 
\firstCell{stereotypeEBackgroundColor} & \footnotesize{\#EB937F} &
\cellcolor[HTML]{EB937F} &
\footnotesize{Background of enum spots in class diagrams}
 
\\ \hline
 
\firstCell{componentArrowColor} & \footnotesize{\#A80036} &
\cellcolor[HTML]{A80036} &
\footnotesize{Color of arrows in component diagrams}
 
\\ \hline
 
\firstCell{componentBackgroundColor} & \footnotesize{\#FEFECE} &
\cellcolor[HTML]{FEFECE} &
\footnotesize{Background of components}
 
\\ \hline
 
\firstCell{componentBorderColor} & \footnotesize{\#A80036} &
\cellcolor[HTML]{A80036} &
\footnotesize{Borders of components}
 
\\ \hline
 
\firstCell{componentInterfaceBackgroundColor} & \footnotesize{\#FEFECE} &
\cellcolor[HTML]{FEFECE} &
\footnotesize{Background of interface in component diagrams}
 
\\ \hline
 
\firstCell{componentInterfaceBorderColor} & \footnotesize{\#A80036} &
\cellcolor[HTML]{A80036} &
\footnotesize{Border of interface in component diagrams}
 
\\ \hline
 
\firstCell{noteBackgroundColor} & \footnotesize{\#FBFB77} &
\cellcolor[HTML]{FBFB77} &
\footnotesize{Background of notes}
 
\\ \hline
 
\firstCell{noteBorderColor} & \footnotesize{\#A80036} &
\cellcolor[HTML]{A80036} &
\footnotesize{Border of notes}
 
\\ \hline
 
\firstCell{stateBackgroundColor} & \footnotesize{\#FEFECE} &
\cellcolor[HTML]{FEFECE} &
\footnotesize{Background of states in state diagrams}
 
\\ \hline
 
\firstCell{stateBorderColor} & \footnotesize{\#A80036} &
\cellcolor[HTML]{A80036} &
\footnotesize{Border of states in state diagrams} 

\\ \hline
 
\firstCell{stateArrowColor} & \footnotesize{\#A80036} &
\cellcolor[HTML]{A80036} &
\footnotesize{Colors of arrows in state diagrams}
 
\\ \hline
 
\firstCell{stateStartColor} & \footnotesize{black} &
\cellcolor[HTML]{000000} &
\footnotesize{Starting circle in state diagrams}
 
\\ \hline
 
\firstCell{stateEndColor} & \footnotesize{black} &
\cellcolor[HTML]{000000} &
\footnotesize{Ending circle in state diagrams}

\\ \hline
 
\firstCell{sequenceArrowColor} & \footnotesize{\#A80036} &
\cellcolor[HTML]{A80036} &
\footnotesize{Color of arrows in sequence diagrams}
 
\\ \hline
 
\firstCell{sequenceActorBackgroundColor} & \footnotesize{\#FEFECE} &
\cellcolor[HTML]{FEFECE} &
\footnotesize{Head's color of actor in sequence diagrams}
 
\\ \hline
 
\firstCell{sequenceActorBorderColor} & \footnotesize{\#A80036} &
\cellcolor[HTML]{A80036} &
\footnotesize{Border of actor in sequence diagrams}

\\ \hline
 
\firstCell{sequenceGroupBackgroundColor} & \footnotesize{\#EEEEEE} &
\cellcolor[HTML]{EEEEEE} &
\footnotesize{Header color of alt/opt/loop in sequence diagrams}
 
\\ \hline
 
\firstCell{sequenceLifeLineBackgroundColor} & \footnotesize{white} &
\cellcolor[HTML]{FFFFFF} &
\footnotesize{Background of life line in sequence diagrams}
 
\\ \hline
 
\firstCell{sequenceLifeLineBorderColor} & \footnotesize{\#A80036} &
\cellcolor[HTML]{A80036} &
\footnotesize{Border of life line in sequence diagrams}
 
\\ \hline
 
\firstCell{sequenceParticipantBackgroundColor} & \footnotesize{\#FEFECE} &
\cellcolor[HTML]{FEFECE} &
\footnotesize{Background of participant in sequence diagrams}
 
\\ \hline
 
\firstCell{sequenceParticipantBorderColor} & \footnotesize{\#A80036} &
\cellcolor[HTML]{A80036} &
\footnotesize{Border of participant in sequence diagrams}
 
\\ \hline
 
\end{tabular}
\end{center}
 
% ========================================================================
\newpage \subsection{Font color, name and size} 

You can change the font for the drawing using \texttt{xxxFontColor},
\texttt{xxxFontSize} and \texttt{xxxFontName} parameters. 

Example: 

\begin{lstlisting}
skinparam classFontColor red
skinparam classFontSize 10
skinparam classFontName Aapex 
\end{lstlisting}


You can also change the default font for all fonts using
\texttt{skinparam defaultFontName}.

Example: 

\begin{lstlisting}
skinparam defaultFontName Aapex
\end{lstlisting}


Please note the fontname is highly system dependant, so do not over use it, if you look for 
portability. 

\newcommand{\firstCellB}{\texttt}

\begin{center}
\begin{longtable}{|l|c|l|} \hline
\textbf{Parameter} & \textbf{Default} & \textbf{Comment}
\\
\textbf{Name} & \textbf{Value} &

\\ \hline

\firstCellB{activityFontColor} & black &
\multirow{4}{*}{Used for activity box} \\*
\firstCellB{activityFontSize}  & 14 &  \\*
\firstCellB{activityFontStyle} & plain & \\*
\firstCellB{activityFontName} & & \\

\hline
\firstCellB{activityArrowFontColor} & black &
\multirow{4}{*}{Used for text on arrows in activity diagrams} \\*
\firstCellB{activityArrowFontSize}  & 13 &  \\*
\firstCellB{activityArrowFontStyle}  & plain & \\*
\firstCellB{activityArrowFontName}  & & \\

\hline
\firstCellB{circledCharacterFontColor} & black &
\multirow{5}{*}{Used for text in circle for class, enum and others} \\*
\firstCellB{circledCharacterFontSize}  & 17 &  \\*
\firstCellB{circledCharacterFontStyle }  & bold & \\*
\firstCellB{circledCharacterFontName}  & Courier & \\*
\firstCellB{circledCharacterRadius}  & 11 & \\

\hline
\firstCellB{classArrowFontColor} & black &
\multirow{4}{*}{Used for text on arrows in class diagrams} \\*
\firstCellB{classArrowFontSize}  & 10 &  \\*
\firstCellB{classArrowFontStyle}  & plain &  \\*
\firstCellB{classArrowFontName}  & & \\

\hline
\firstCellB{classAttributeFontColor}  & black &
\multirow{4}{*}{Class attributes and methods} \\*
\firstCellB{classAttributeFontSize}  & 10 &  \\*
\firstCellB{classAttributeIconSize}  & 10 &  \\*
\firstCellB{classAttributeFontStyle}  & plain &  \\*
\firstCellB{classAttributeFontName}  & & \\

\hline
\firstCellB{classFontColor}  & black &
\multirow{4}{*}{Used for classes name} \\*
\firstCellB{classFontSize}  & 12 &  \\*
\firstCellB{classFontStyle}  & plain &  \\*
\firstCellB{classFontName}  & & \\

\hline
\firstCellB{classStereotypeFontColor}  & black &
\multirow{4}{*}{Used for stereotype in classes} \\*
\firstCellB{classStereotypeFontSize}  & 12 &  \\*
\firstCellB{classStereotypeFontStyle}  & italic &  \\*
\firstCellB{classStereotypeFontName}  & & \\

\hline
\firstCellB{componentFontColor}  & black &
\multirow{4}{*}{Used for components name} \\*
\firstCellB{componentFontSize}  & 14 &  \\*
\firstCellB{componentFontStyle}  & plain &  \\*
\firstCellB{componentFontName}  & & \\

\hline
\firstCellB{componentStereotypeFontColor}  & black &
\multirow{4}{*}{Used for stereotype in components} \\*
\firstCellB{componentStereotypeFontSize}  & 14 &  \\*
\firstCellB{componentStereotypeFontStyle}  & italic &  \\*
\firstCellB{componentStereotypeFontName}  & & \\

\hline
\firstCellB{componentArrowFontColor}  & black &
\multirow{4}{*}{Used for text on arrows in component diagrams} \\*
\firstCellB{componentArrowFontSize}  & 13 &  \\*
\firstCellB{componentArrowFontStyle}  & plain &  \\*
\firstCellB{componentArrowFontName}  & & \\

\hline
\firstCellB{noteFontColor}  & black &
\multirow{4}{*}{Used for notes in all diagrams but sequence diagrams} \\*
\firstCellB{noteFontSize}  & 13 &  \\*
\firstCellB{noteFontStyle}  & plain &  \\*
\firstCellB{noteFontName}  & & \\

\hline
\firstCellB{packageFontColor}  & black &
\multirow{4}{*}{Used for package and partition names} \\*
\firstCellB{packageFontSize}  & 14 &  \\*
\firstCellB{packageFontStyle}  & plain &  \\*
\firstCellB{packageFontName}  & & \\

\hline
\firstCellB{sequenceActorFontColor}  & black &
\multirow{4}{*}{Used for actor in sequence diagrams} \\*
\firstCellB{sequenceActorFontSize}   & 13 &  \\*
\firstCellB{sequenceActorFontStyle}   & plain &  \\*
\firstCellB{sequenceActorFontName}   & & \\

\hline
\firstCellB{sequenceDividerFontColor}  & black &
\multirow{4}{*}{Used for text on dividers in sequence diagrams} \\*
\firstCellB{sequenceDividerFontSize}   & 13 &  \\*
\firstCellB{sequenceDividerFontStyle}   & bold &  \\*
\firstCellB{sequenceDividerFontName}   & & \\

\hline
\firstCellB{sequenceArrowFontColor}   & black &
\multirow{4}{*}{Used for text on arrows in sequence diagrams} \\*
\firstCellB{sequenceArrowFontSize}   & 13 &  \\*
\firstCellB{sequenceArrowFontStyle}   & plain &  \\*
\firstCellB{sequenceArrowFontName}   & & \\

\hline
\firstCellB{sequenceGroupingFontColor}   & black &
\multirow{4}{*}{Used for text for "else" in sequence diagrams} \\*
\firstCellB{sequenceGroupingFontSize}   & 11 &  \\*
\firstCellB{sequenceGroupingFontStyle}   & plain &  \\*
\firstCellB{sequenceGroupingFontName}   & & \\

\hline
\firstCellB{sequenceGroupingHeaderFontColor}   & black &
\multirow{4}{*}{Used for text for "alt/opt/loop" headers in sequence diagrams} \\*
\firstCellB{sequenceGroupingHeaderFontSize}   & 13 &  \\*
\firstCellB{sequenceGroupingHeaderFontStyle}   & plain &  \\*
\firstCellB{sequenceGroupingHeaderFontName}   & & \\

\hline
\firstCellB{sequenceParticipantFontColor}   & black &
\multirow{4}{*}{Used for text on participant in sequence diagrams} \\*
\firstCellB{sequenceParticipantFontSize}   & 13 &  \\*
\firstCellB{sequenceParticipantFontStyle}   & plain &  \\*
\firstCellB{sequenceParticipantFontName}   & & \\

\hline
\firstCellB{sequenceTitleFontColor}   & black &
\multirow{4}{*}{Used for titles in sequence diagrams} \\*
\firstCellB{sequenceTitleFontSize}   & 13 &  \\*
\firstCellB{sequenceTitleFontStyle}   & plain &  \\*
\firstCellB{sequenceTitleFontName}   & & \\

\hline
\firstCellB{titleFontColor}   & black &
\multirow{4}{*}{Used for titles in all diagrams but sequence diagrams} \\* 
\firstCellB{titleFontSize}   & 18 &  \\*
\firstCellB{titleFontStyle}   & plain &  \\*
\firstCellB{titleFontName}   & & \\

\hline
\firstCellB{stateFontColor}   & black &
\multirow{4}{*}{Used for states in state diagrams} \\*
\firstCellB{stateFontSize}   & 14 &  \\*
\firstCellB{stateFontStyle}   & plain &  \\*
\firstCellB{stateFontName}   & & \\

\hline
\firstCellB{stateArrowFontColor}   & black &
\multirow{4}{*}{Used for text on arrows in state diagrams} \\*
\firstCellB{stateArrowFontSize}   & 13 &  \\*
\firstCellB{stateArrowFontStyle}   & plain &  \\*
\firstCellB{stateArrowFontName}   & & \\

\hline
\firstCellB{stateAttributeFontColor}   & black &
\multirow{4}{*}{Used for states description in state diagrams} \\*
\firstCellB{stateAttributeFontSize}   & 12 &  \\*
\firstCellB{stateAttributeFontStyle}   & plain &  \\*
\firstCellB{stateAttributeFontName}   & & \\

\hline
\firstCellB{usecaseFontColor}   & black &
\multirow{4}{*}{Used for usecase labels in usecase diagrams} \\*
\firstCellB{usecaseFontSize}   & 14 &  \\*
\firstCellB{usecaseFontStyle}   & plain &  \\*
\firstCellB{usecaseFontName}   & & \\

\hline
\firstCellB{usecaseStereotypeFontColor}   & black &
\multirow{4}{*}{Used for stereotype in usecase} \\*
\firstCellB{usecaseStereotypeFontSize}   & 14 &  \\*
\firstCellB{usecaseStereotypeFontStyle}   & italic &  \\*
\firstCellB{usecaseStereotypeFontName}   & & \\

\hline
\firstCellB{usecaseActorFontColor}   & black &
\multirow{4}{*}{Used for actor labels in usecase diagrams} \\*
\firstCellB{usecaseActorFontSize}   & 14 &  \\*
\firstCellB{usecaseActorFontStyle}   & plain &  \\*
\firstCellB{usecaseActorFontName}   & & \\

\hline
\firstCellB{usecaseActorStereotypeFontColor}   & black &
\multirow{4}{*}{Used for stereotype for actor} \\*
\firstCellB{usecaseActorStereotypeFontSize}   & 14 &  \\*
\firstCellB{usecaseActorStereotypeFontStyle}   & italic &  \\*
\firstCellB{usecaseActorStereotypeFontName}   & & \\

\hline
\firstCellB{usecaseArrowFontColor}   & black &
\multirow{4}{*}{Used for text on arrows in usecase diagrams} \\*
\firstCellB{usecaseArrowFontSize}   & 13 &  \\*
\firstCellB{usecaseArrowFontStyle}   & plain &  \\*
\firstCellB{usecaseArrowFontName}   & & \\

\hline
\firstCellB{footerFontColor}   & black &
\multirow{4}{*}{Used for footer} \\*
\firstCellB{footerFontSize}   & 10 &  \\*
\firstCellB{footerFontStyle}   & plain &  \\*
\firstCellB{footerFontName}   & & \\

\hline
\firstCellB{headerFontColor}   & black &
\multirow{4}{*}{Used for header} \\*
\firstCellB{headerFontSize}   & 10 &  \\*
\firstCellB{headerFontStyle}   & plain &  \\*
\firstCellB{headerFontName}   & & \\

\hline
\end{longtable}
\end{center}

% ========================================================================
\newpage \subsection{Black and White} 

You can force the use of a black&white output using the
\texttt{skinparam monochrome true} command.
 
\begin{lstlisting}
@startuml
skinparam monochrome true

actor User
participant "First Class" as A
participant "Second Class" as B
participant "Last Class" as C

User -> A: DoWork
activate A

A -> B: Create Request
activate B

B -> C: DoWork
activate C
C --> B: WorkDone
destroy C

B --> A: Request Created
deactivate B

A --> User: Done
deactivate A

@enduml
\end{lstlisting}

\begin{center}
\includegraphics[width=100mm]{skinparam.png}
\end{center}
 