% ========================================================================
% PlantUML : a free UML diagram generator
% ========================================================================
%
% (C) Copyright 2009, Arnaud Roques
%
% Project Info:  http://plantuml.sourceforge.net
% 
% This file is part of PlantUML.
%
% PlantUML is free software; you can redistribute it and/or modify it
% under the terms of the GNU General Public License as published by
% the Free Software Foundation, either version 3 of the License, or
% (at your option) any later version.
%
% PlantUML distributed in the hope that it will be useful, but
% WITHOUT ANY WARRANTY; without even the implied warranty of MERCHANTABILITY
% or FITNESS FOR A PARTICULAR PURPOSE. See the GNU Lesser General Public
% License for more details.
%
% You should have received a copy of the GNU General Public
% License along with this library; if not, write to the Free Software
% Foundation, Inc., 51 Franklin Street, Fifth Floor, Boston, MA  02110-1301,
% USA.
%
% [Java is a trademark or registered trademark of Sun Microsystems, Inc.
% in the United States and other countries.]
%
% Original Author:  Arnaud Roques
% 

\section{Class Diagram}

% ========================================================================
\subsection{Relations between classes}

\begin{description}
\item Relations between classes are defined using the following symbols : 

\begin{tabular}{|l|l|l|} \hline
Extension & \texttt{<|--} & 
\includegraphics[width=6mm]{img/extends01.png}

\\ \hline
Composition & \texttt{*--} &
\includegraphics[width=7mm]{img/sym03.png}

\\ \hline
Agregation & \texttt{o--} &
\includegraphics[width=7mm]{img/sym01.png}

\\ \hline
\end{tabular}

\item It is possible to replace "\texttt{--}" by "\texttt{..}" to have a dotted
line.

\item Knowing thoses rules, it is possible to draw the following drawings: 
\end{description}

\begin{lstlisting}
@startuml
Class01 <|-- Class02
Class03 *-- Class04
Class05 o-- Class06
Class07 .. Class08
Class09 -- Class10
Class11 <|.. Class12
Class13 --> Class14
Class15 ..> Class16
Class17 ..|> Class18
Class19 <--* Class20
@enduml
\end{lstlisting}
\begin{center}
\includegraphics[width=170mm]{class.png}
\end{center}

% ========================================================================
\newpage \subsection{Label on relations}

\begin{description}
\item It is possible a add a label on the relation, using "\texttt{:}", followed
by the text of the label.
\item For cardinality, you can use double-quotes \texttt{""} on each side
of the relation.
\end{description}

\begin{lstlisting}
@startuml

Class01 "1" *-- "many" Class02 : contains

Class03 o-- Class04 : agregation

Class05 --> "1" Class06

@enduml
\end{lstlisting}
\begin{center}
\includegraphics[width=80mm]{class_001.png}
\end{center}


% ========================================================================
\newpage \subsection{Adding methods}

\begin{description}
\item To declare fields and methods, you can use the symbol "\texttt{:}"
followed by the field's or method's name.
\item The system checks for parenthesis to choose between methods and fields.
\end{description}

\begin{lstlisting}
@startuml
Object <|-- ArrayList

Object : equals()
ArrayList : Object[] elementData
ArrayList : size()

@enduml
\end{lstlisting}
\begin{center}
\includegraphics[width=25mm]{class_002.png}
\end{center}

\begin{description}
\item It is also possible to group between brackets \texttt{\{\}}
all fields and methods.
\end{description}

\begin{lstlisting}
@startuml
class Dummy {
  String data
  void methods()
}
@enduml
\end{lstlisting}
\begin{center}
\includegraphics[width=25mm]{class_003.png}
\end{center}


% ========================================================================
\newpage \subsection{Defining visibility}

\begin{description}
\item When you define methods or fields, you can use characters to define the
visibility of the corresponding item:
\end{description}

\begin{tabular}{|c|c|c|l|} \hline
\texttt{-} & 
\includegraphics[width=2mm]{img/PRIVATE_FIELD.png} &
\includegraphics[width=2mm]{img/PRIVATE_METHOD.png} &
\texttt{private}
\\ \hline

\texttt{\#} & 
\includegraphics[width=2mm]{img/PROTECTED_FIELD.png} &
\includegraphics[width=2mm]{img/PROTECTED_METHOD.png} &
\texttt{protected}
\\ \hline

\texttt{\~} & 
\includegraphics[width=2mm]{img/PACKAGE_PRIVATE_FIELD.png} &
\includegraphics[width=2mm]{img/PACKAGE_PRIVATE_METHOD.png} &
\texttt{package private}
\\ \hline

\texttt{+} & 
\includegraphics[width=2mm]{img/PUBLIC_FIELD.png} &
\includegraphics[width=2mm]{img/PUBLIC_METHOD.png} &
\texttt{public}
\\ \hline
\end{tabular}

\vskip 10mm

\begin{lstlisting}
@startuml
class Dummy {
  -field1
  #field2
  ~method1()
  +method2()
}
@enduml
\end{lstlisting}

\begin{center}
\includegraphics[width=20mm]{class_004.png}
\end{center}

\vskip 10mm

\begin{description}
\item You can turn off this feature using the
\texttt{skinparam classAttributeIconSize 0} command :
\end{description}

\begin{lstlisting}
@startuml
skinparam classAttributeIconSize 0
class Dummy {
  -field1
  #field2
  ~method1()
  +method2()
}
@enduml
\end{lstlisting}
\begin{center}
\includegraphics[width=20mm]{class_005.png}
\end{center}


% ========================================================================
\newpage \subsection{Notes and stereotypes}

\begin{description}
\item Stereotypes are defined with the \texttt{class} keyword, "\texttt{<<}" and
"\texttt{>>}".
\item You can alse define notes using \texttt{note left of},
\texttt{note right of}, \texttt{note top of}, \texttt{note bottom of} keywords.
\item A note can be also define alone with the \texttt{note} keywords, then linked to
other objects using the "\texttt{..}" symbol.
\end{description}

\begin{lstlisting}
@startuml
class Object << general >>
Object <|--- ArrayList

note top of Object : In java, every class\nextends this one.

note "This is a floating note" as N1
note "This note is connected\nto several objects." as N2
Object .. N2
N2 .. ArrayList

@enduml
\end{lstlisting}
\begin{center}
\includegraphics[width=70mm]{class_006.png}
\end{center}

% ========================================================================
\newpage \subsection{More on notes}

It is also possible to use few html tags like :
\begin{itemize}
\item \texttt{$<$b$>$}
\item \texttt{$<$u$>$}
\item \texttt{$<$i$>$}
\item \texttt{$<$s$>$}, \texttt{$<$del$>$}, \texttt{$<$strike$>$}
\item \texttt{$<$font color="\#AAAAAA"$>$} or \texttt{$<$font
color="colorName"$>$}
\item \texttt{$<$color:\#AAAAAA$>$} or \texttt{$<$color:colorName$>$}
\item \texttt{$<$size:nn$>$} to change font size
\item \texttt{$<$img src="file"$>$} or \texttt{$<$img:file$>$} : the file
must be accessible by the filesystem
\end{itemize}

You can also have a note on several lines.

\begin{lstlisting}
@startuml

note top of Object
  In java, every <u>class</u>
  <b>extends</b>
  <i>this</i> one.
end note

note as N1
  This <size:10>note</size> is <u>also</u>
  <b><color:royalBlue>on several</color>
  <s>words</s> lines
  And this is hosted by <img:sourceforge.jpg>
end note

@enduml
\end{lstlisting}
\begin{center}
\includegraphics[width=90mm]{class_007.png}
\end{center}

% ========================================================================
\newpage \subsection{Abstract class and interface}

You can declare a class as abstract using "\texttt{abstract}" or
"\texttt{abstract class}" keywords. The class will be printed in italic.

You can use the \texttt{interface} and \texttt{enum} keywords too.

\begin{lstlisting}
@startuml

abstract class AbstractList
abstract AbstractCollection
interface List
interface Collection

List <|-- AbstractList
Collection <|-- AbstractCollection

Collection <|- List
AbstractCollection <|- AbstractList
AbstractList <|-- ArrayList

ArrayList : Object[] elementData
ArrayList : size()

enum TimeUnit
TimeUnit : DAYS
TimeUnit : HOURS
TimeUnit : MINUTES

@enduml
\end{lstlisting}
\begin{center}
\includegraphics[width=80mm]{class_008.png}
\end{center}

% ========================================================================
\newpage \subsection{Using non-letters}

If you want to use non-letters in the class (or enum...) display, you can either :

\begin{itemize}
  \item Use the as keyword in the class definition
  \item Put quotes \texttt{""} around the class name
\end{itemize}

\begin{lstlisting}
@startuml
class "This is my class" as class1
class class2 as "It works this way too"

class2 *-- "foo/dummy" : use
@enduml
\end{lstlisting}
\begin{center}
\includegraphics[width=80mm]{class_009.png}
\end{center}


% ========================================================================
\newpage \subsection{Hide attributes, methods...}

\begin{description}
\item You can parameterize the display of classes using the
\texttt{hide}/\texttt{show} command.

\item The basic command is: \texttt{hide empty members}. This command will hide
attributes or methods if they are empty.

\item Instead of \texttt{empty members}, you can use:
\end{description}

\begin{itemize}
    \item \texttt{empty fields} or \texttt{empty attributes} for empty fields,
    \item \texttt{empty methods} for empty methods,
    \item \texttt{fields} or \texttt{attributes} which will hide fields, even if
    they are described,
    \item \texttt{methods} wich will hide methods, even if they are described,
    \item \texttt{members} wich will hide fields and methods, even if they are
    described,
    \item \texttt{circle} for the circled character in front of class name,
    \item \texttt{stereotype} for the stereotype. 
\end{itemize}

You can also provide, just after the \texttt{hide} or \texttt{show} keyword:

\begin{itemize}
    \item  \texttt{class} for all classes,
    \item  \texttt{interface} for all interfaces,
    \item  \texttt{enum} for all enums,
    \item  \texttt{<<foo1>>} for classes which are stereotyped with
    \textit{foo1},
    \item  an existing class name.
\end{itemize}

You can use several \texttt{show}/\texttt{hide} commands to define rules and
exceptions.

\begin{lstlisting}
@startuml
class Dummy1 {
  +myMethods()
}

class Dummy2 {
  +hiddenMethod()
}

class Dummy3 <<Serializable>> {
	String name
}

hide members
hide <<Serializable>> circle
show Dummy1 method
show <<Serializable>> fields
@enduml
\end{lstlisting}
\begin{center}
\includegraphics[width=90mm]{class_010.png}
\end{center}


% ========================================================================
\newpage \subsection{Specific Spot}

Usually, a spotted character (C, I, E or A) is used for classes, interface, enum and abstract classes.
But you can define your own spot for a class when you define the stereotype, adding a single character
and a color, like in this example:

\begin{lstlisting}
@startuml

class System << (S,#FF7700) Singleton >>
class Date << (D,orchid) >>
@enduml
\end{lstlisting}
\begin{center}
\includegraphics[width=60mm]{class_011.png}
\end{center}
		
% ========================================================================
\newpage \subsection{Packages}

You can define a package using the \texttt{package} keyword, and optionally
declare a background color for your package (Using a html color code or name).
When you declare classes, they are automatically put in the last used package,
and you can close the package definition using the \texttt{end package} keyword.
You can also use brackets \{ \}.

Note that package definitions can be nested.

\begin{lstlisting}
@startuml

package "Classic Collections" #DDDDDD {
  Object <|-- ArrayList
}

package net.sourceforge.plantuml #Snow
  Object <|-- Demo1
  Demo1 *- Demo2
end package

@enduml
\end{lstlisting}
\begin{center}
\includegraphics[width=80mm]{class_012.png}
\end{center}
		

You can also define links between packages, like in the following example:
\begin{lstlisting}
@startuml

package foo1.foo2
end package

package foo1.foo2.foo3 {
  class Object
}

foo1.foo2 +-- foo1.foo2.foo3

@enduml
\end{lstlisting}
\begin{center}
\includegraphics[width=50mm]{class_013.png}
\end{center}
		
		
% ========================================================================
\newpage \subsection{Namespaces}

In packages, the name of a class is the unique identifier of this class.
It means that you cannot have two classes with the very same name in different packages.
In that case, you should use namespaces instead of packages.

You can refer to classes from other namespaces by fully qualify them.
Classes from the default namespace are qualified with a starting dot.

Note that you don't have to explicitly create namespace : a fully qualified class
is automatically put in the right namespace.

\begin{lstlisting}
@startuml

class BaseClass

namespace net.dummy #DDDDDD
    .BaseClass <|-- Person
    Meeting o-- Person
    
    .BaseClass <|- Meeting

end namespace

namespace net.foo {
  net.dummy.Person  <|- Person
  .BaseClass <|-- Person

  net.dummy.Meeting o-- Person
}

BaseClass <|-- net.unused.Person

@enduml
\end{lstlisting}
\begin{center}
\includegraphics[width=85mm]{class_014.png}
\end{center}

% ========================================================================
\newpage \subsection{Changing arrows direction}

By default, links between classes have two dashes \texttt{--} and are verticaly
oriented. It is possible to use horizontal link by putting a single dash (or dot) like this:

\begin{lstlisting}
@startuml
Room o- Studient
Room *-- Chair
@enduml
\end{lstlisting}
\begin{center}
\includegraphics[width=40mm]{class_015.png}
\end{center}

You can also change directions by reversing the link:
\begin{lstlisting}
@startuml
Studient -o Room
Chair --* Room
@enduml
\end{lstlisting}
\begin{center}
\includegraphics[width=40mm]{class_016.png}
\end{center}

It is also possible to change arrow direction by adding \texttt{left},
\texttt{right}, \texttt{up} or \texttt{down} keywords inside the arrow:

\begin{lstlisting}
@startuml
foo -left-> dummyLeft 
foo -right-> dummyRight 
foo -up-> dummyUp 
foo -down-> dummyDown
@enduml
\end{lstlisting}
\begin{center}
\includegraphics[width=70mm]{class_017.png}
\end{center}

You can shorten the arrow by using only the first character of the direction
(for example, \texttt{-d-} instead of \texttt{-down-}) or the two first
characters (\texttt{-do-})

Please note that you should not abuse this functionnality : \textit{GraphViz}
gives usually good results without tweaking.

% ========================================================================
\subsection{Lollipop interface}

You can also define lollipops interface on classes, using the following syntax:

\begin{itemize}
    \item  \texttt{bar ()- foo},
    \item  \texttt{bar ()-- foo},
    \item  \texttt{foo -() bar}
\end{itemize}

\begin{lstlisting}
@startuml
class foo
bar ()- foo
@enduml
\end{lstlisting}


\begin{center}
\includegraphics[width=40mm]{class_018.png}
\end{center}

% ========================================================================
\subsection{Title the diagram}

The title \texttt{keywords} is used to put a title.
You can use title and end title keywords for a longer title, as in sequence diagrams.

\begin{lstlisting}
@startuml
title Simple <b>example</b>\nof title 

Object <|-- ArrayList

@enduml
\end{lstlisting}
\begin{center}
\includegraphics[width=25mm]{class_019.png}
\end{center}

% ========================================================================
\newpage \subsection{Association classes}

You can define \textit{association class} after that a relation has been defined
between two classes, like in this example:

\begin{lstlisting}
@startuml
Student : Name
Student "0..*" - "1..*" Course
(Student, Course) .. Enrollment

Enrollment : drop()
Enrollment : cancel()
@enduml
\end{lstlisting}
\begin{center}
\includegraphics[width=50mm]{class_020.png}
\end{center}

You can define it in another direction:

\begin{lstlisting}
@startuml
Student : Name
Student "0..*" -- "1..*" Course
(Student, Course) . Enrollment

Enrollment : drop()
Enrollment : cancel()
@enduml
\end{lstlisting}
\begin{center}
\includegraphics[width=35mm]{class_021.png}
\end{center}

% ========================================================================
\newpage \subsection{Skinparam}

You can use the \texttt{skinparam} command to change colors and fonts for the
drawing. You can use this command :

\begin{itemize}
  \item In the diagram definition, like any other commands,
  \item In an included file,
  \item In a configuration file, provided in the command line or the ANT task.
\end{itemize}

\begin{lstlisting}
@startuml

skinparam classBackgroundColor PaleGreen
skinparam classArrowColor SeaGreen
skinparam classBorderColor SpringGreen
skinparam stereotypeCBackgroundColor YellowGreen

Class01 "1" *-- "many" Class02 : contains

Class03 o-- Class04 : agregation

@enduml
\end{lstlisting}
\begin{center}
\includegraphics[width=50mm]{class_022.png}
\end{center}

% ========================================================================
\newpage \subsection{Skinned Stereotypes}
You can define specific color and fonts for stereotyped classes.

\begin{lstlisting}
@startuml

skinparam class {
	BackgroundColor PaleGreen
	ArrowColor SeaGreen
	BorderColor SpringGreen
	BackgroundColor<<Foo>> Wheat
	BorderColor<<Foo>> Tomato
}
skinparam stereotypeCBackgroundColor YellowGreen
skinparam stereotypeCBackgroundColor<<Foo>> DimGray

Class01 <<Foo>>
Class01 "1" *-- "many" Class02 : contains

Class03 <<Foo>> o-- Class04 : agregation

@enduml
\end{lstlisting}
\begin{center}
\includegraphics[width=60mm]{class_023.png}
\end{center}


% ========================================================================
\newpage \subsection{Splitting large files}

Sometimes, you will get some very large image files.
You can use the "\texttt{page (hpages)x(vpages)}" command to split the generated
image into several files :
\begin{itemize}
  \item \textit{hpages} is a number that indicated the number of horizontal
  pages,
  \item \textit{vpages} is a number that indicated the number of vertical pages. 
\end{itemize}


\begin{lstlisting}
@startuml
' Split into 4 pages
page 2x2

class BaseClass

namespace net.dummy #DDDDDD
    .BaseClass <|-- Person
    Meeting o-- Person
    
    .BaseClass <|- Meeting

end namespace

namespace net.foo {
  net.dummy.Person  <|- Person
  .BaseClass <|-- Person

  net.dummy.Meeting o-- Person
}

BaseClass <|-- net.unused.Person
@enduml
\end{lstlisting}
		
\begin{center}
	\fbox{\includegraphics[width=45mm]{class_024.png}}
	\hspace*{4mm}
	\fbox{\includegraphics[width=45mm]{class_024_002.png}}
	\par
	\vskip 4mm
	\fbox{\includegraphics[width=45mm]{class_024_001.png}}
	\hspace*{4mm}
	\fbox{\includegraphics[width=45mm]{class_024_003.png}}
\end{center}
	