% ========================================================================
% PlantUML : a free UML diagram generator
% ========================================================================
%
% (C) Copyright 2009, Arnaud Roques
%
% Project Info:  http://plantuml.sourceforge.net
% 
% This file is part of PlantUML.
%
% PlantUML is free software; you can redistribute it and/or modify it
% under the terms of the GNU General Public License as published by
% the Free Software Foundation, either version 3 of the License, or
% (at your option) any later version.
%
% PlantUML distributed in the hope that it will be useful, but
% WITHOUT ANY WARRANTY; without even the implied warranty of MERCHANTABILITY
% or FITNESS FOR A PARTICULAR PURPOSE. See the GNU Lesser General Public
% License for more details.
%
% You should have received a copy of the GNU General Public
% License along with this library; if not, write to the Free Software
% Foundation, Inc., 51 Franklin Street, Fifth Floor, Boston, MA  02110-1301,
% USA.
%
% [Java is a trademark or registered trademark of Sun Microsystems, Inc.
% in the United States and other countries.]
%
% Original Author:  Arnaud Roques
% 

\section{Preprocessing}


Some minor preprocessing capabilities are included in PlantUML, and available for all diagrams. 

Thoses functionnalities are very similar to the \textbf{C} language
preprocessor, except that the special character "\texttt{\#}" has been changed
to the exclamation mark "\texttt{!}".

% ========================================================================
\subsection{Including files} 

Use the \texttt{!include} directive to include file in your diagram. 

Imagine you have the very same class that appears in many diagrams. Instead of duplicating the 
description of this class, you can define a file that contains the description. 

\begin{lstlisting}
@startuml
!include List.iuml 
List <|.. ArrayList
@enduml 
\end{lstlisting}
\begin{center}
\includegraphics[width=25mm]{preprocessing.png}
\end{center}


File List.iuml: 
\begin{lstlisting}
interface List 
List : int size()
List : void clear() 
\end{lstlisting}


The file \texttt{List.iuml} can be included in many diagrams, and any
modification in this file will change all diagrams that include it. 

% ========================================================================
\newpage \subsection{Constant definition} 

You can define constant using the \texttt{!define} directive. As in \textbf{C}
language, a constant name can only use alphanumeric and underscore characters, and cannot start with a digit. 

\begin{lstlisting}
@startuml 

!define SEQUENCE (S,#AAAAAA) Database Sequence
!define TABLE (T,#FFAAAA) Database Table 


class USER << TABLE >> 
class ACCOUNT << TABLE >> 
class UID << SEQUENCE >>
USER "1" -- "*" ACCOUNT 
USER -> UID 

@enduml 
\end{lstlisting}
\begin{center}
\includegraphics[width=80mm]{preprocessing_001.png}
\end{center}



Of course, you can use the \texttt{!include} directive to define all your
constants in a single file that you include in your diagram. 

Constant can be undefined with the \texttt{!undef} XXX directive. 

% ========================================================================
\newpage \subsection{Conditions} 

You can use \texttt{!ifdef XXX} and \texttt{!endif} directives to have
conditionnal drawings.

The lines between those two directives will be included only if the constant
after the \texttt{!ifdef} directive has been defined before. 

You can also provide a \texttt{!else} part which will be included if the
constant has not been defined.

\begin{lstlisting}
@startuml
!include ArrayList.iuml
@enduml 
\end{lstlisting}

\begin{center}
\includegraphics[width=25mm]{preprocessing_002.png}
\end{center}


File ArrayList.iuml: 
\begin{lstlisting}
class ArrayList
!ifdef SHOW_METHODS 
ArrayList : int size()
ArrayList : void clear()
!endif 
\end{lstlisting}



You can then use the \texttt{!define} directive to activate the conditionnal
part of the diagram.

\begin{lstlisting}
@startuml
!define SHOW_METHODS 
!include ArrayList.iuml
@enduml 
\end{lstlisting}

\begin{center}
\includegraphics[width=25mm]{preprocessing_003.png}
\end{center}


You can also use the \texttt{!ifndef} directive that includes lines if the
provided constant has \textit{NOT} been defined. 

