% ========================================================================
% PlantUML : a free UML diagram generator
% ========================================================================
%
% (C) Copyright 2009, Arnaud Roques
%
% Project Info:  http://plantuml.sourceforge.net
% 
% This file is part of PlantUML.
%
% PlantUML is free software; you can redistribute it and/or modify it
% under the terms of the GNU General Public License as published by
% the Free Software Foundation, either version 3 of the License, or
% (at your option) any later version.
%
% PlantUML distributed in the hope that it will be useful, but
% WITHOUT ANY WARRANTY; without even the implied warranty of MERCHANTABILITY
% or FITNESS FOR A PARTICULAR PURPOSE. See the GNU Lesser General Public
% License for more details.
%
% You should have received a copy of the GNU General Public
% License along with this library; if not, write to the Free Software
% Foundation, Inc., 51 Franklin Street, Fifth Floor, Boston, MA  02110-1301,
% USA.
%
% [Java is a trademark or registered trademark of Sun Microsystems, Inc.
% in the United States and other countries.]
%
% Original Author:  Arnaud Roques
% 

\section{State Diagram}

% ========================================================================
\subsection{Simple State}

\begin{description}
\item You can use \texttt{[*]} for the starting point and ending point of the
state diagram.
\item Use \texttt{-->} for arrows.
\end{description}

\begin{lstlisting}
@startuml
[*] --> State1
State1 --> [*]
State1 : this is a string
State1 : this is another string

State1 -> State2
State2 --> [*]
@enduml
\end{lstlisting}

\begin{center}
\includegraphics[width=60mm]{state.png}
\end{center}


% ========================================================================
\newpage \subsection{Composite state}

A state can also be composite. You have to define it using the \texttt{state}
keywords and brackets.


\begin{lstlisting}
@startuml

[*] --> NotShooting

state NotShooting {
  [*] --> Idle
  Idle --> Configuring : EvConfig
  Configuring --> Idle : EvConfig
}

state Configuring {
  [*] --> NewValueSelection
  NewValueSelection --> NewValuePreview : EvNewValue
  NewValuePreview --> NewValueSelection : EvNewValueRejected
  NewValuePreview --> NewValueSelection : EvNewValueSaved
  
  state NewValuePreview {
     State1 -> State2
  }
  
}
@enduml
\end{lstlisting}

\begin{center}
\includegraphics[width=80mm]{state_001.png}
\end{center}


% ========================================================================
\newpage \subsection{Long name}

You can also use the \texttt{state} keyword to use long description for states.

\begin{lstlisting}
@startuml

[*] -> State1
State1 --> State2 : Succeeded
State1 --> [*] : Aborted
State2 --> State3 : Succeeded
State2 --> [*] : Aborted
state State3 {
  state "Accumulate Enough Data\nLong State Name" as long1
  long1 : Just a test
  [*] --> long1
  long1 --> long1 : New Data
  long1 --> ProcessData : Enough Data
}
State3 --> State3 : Failed
State3 --> [*] : Succeeded / Save Result
State3 --> [*] : Aborted
 
@enduml
\end{lstlisting}

\begin{center}
\includegraphics[width=110mm]{state_002.png}
\end{center}

% ========================================================================
\newpage \subsection{Concurrent state}

You can define concurrent state into a composite state using the "\texttt{--}"
symbol as separator.

\begin{lstlisting}
@startuml

[*] --> Active

state Active {
  [*] -> NumLockOff
  NumLockOff --> NumLockOn : EvNumLockPressed
  NumLockOn --> NumLockOff : EvNumLockPressed
  --
  [*] -> CapsLockOff
  CapsLockOff --> CapsLockOn : EvCapsLockPressed
  CapsLockOn --> CapsLockOff : EvCapsLockPressed
  --
  [*] -> ScrollLockOff
  ScrollLockOff --> ScrollLockOn : EvCapsLockPressed
  ScrollLockOn --> ScrollLockOff : EvCapsLockPressed
} 

@enduml
\end{lstlisting}
\begin{center}
\includegraphics[width=150mm]{state_003.png}
\end{center}
        
% ========================================================================
\newpage \subsection{Arrow direction}

You can use \texttt{->} for horizontal arrows. It is possible to force arrow's
direction using the following syntax:

\begin{itemize}
\item \texttt{-down->} (default arrow)
\item \texttt{-right->} or \texttt{->}
\item \texttt{-left->}
\item \texttt{-up->}
\end{itemize}

\begin{lstlisting}
@startuml

[*] -up-> First
First -right-> Second
Second --> Third
Third -left-> Last

@enduml
\end{lstlisting}
\begin{center}
\includegraphics[width=50mm]{state_004.png}
\end{center}
        
You can shorten the arrow by using only the first character of the direction
(for example, \texttt{-d-} instead of \texttt{-down-}) or the two first
characters (\texttt{-do-}).

Please note that you should not abuse this functionnality : \textit{GraphViz}
gives usually good results without tweaking.

% ========================================================================
\newpage \subsection{Note}

You can alse define notes using:
\begin{itemize}
  \item \texttt{note left of},
  \item \texttt{note right of},
  \item \texttt{note top of},
  \item \texttt{note bottom of}  
\end{itemize}
keywords. You can also define notes on several lines.

\begin{lstlisting}
@startuml

[*] --> Active
Active --> Inactive

note left of Active : this is a short\nnote

note right of Inactive
  A note can also
  be defined on
  several lines
end note

@enduml
\end{lstlisting}
\begin{center}
\includegraphics[width=90mm]{state_005.png}
\end{center}


% ========================================================================
\newpage \subsection{More in notes}

You can put notes on composite states.

\begin{lstlisting}
@startuml

[*] --> NotShooting

state "Not Shooting State" as NotShooting {
  state "Idle mode" as Idle
  state "Configuring mode" as Configuring
  [*] --> Idle
  Idle --> Configuring : EvConfig
  Configuring --> Idle : EvConfig
}

note right of NotShooting : This is a note on a composite state

@enduml
\end{lstlisting}
\begin{center}
\includegraphics[width=90mm]{state_006.png}
\end{center}
        

        

        
